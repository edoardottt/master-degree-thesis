\begin{abstract}
The thesis focuses on the study of a framework for the detection of Cross Application Poisoning attacks in Software Defined Networks. In this type of attacks, malicious applications do not have direct access to the functions that would allow them to carry out actions that would give benefit to the attacker. Security mechanisms are implemented, but malicious applications still manage to inject malicious payloads using the functions they have access to and manage to induce other legitimate applications to carry out the attack on their behalf. The study begins by analyzing existing state-of-the-art solutions and highlighting their limitations. A study is also carried out on the architecture of ONOS, which is the controller for Software Defined networks considered in this research activity. Three implementations of Cross Application Poisoning attacks with different goals are presented, these will also be used in tests for detection. The framework implements an additional data store for registering applications that want to use the APIs exposed by the controller. The APIs require login credentials and every action is logged in a centralized repository. An analysis is then performed using the log file to build a graph representing all interactions between the controller and applications looking for Cross Application Poisoning attack patterns. An in-depth study is also carried out on the security of the proposed solution and on the various attacks that could endanger the components that have been added and the entire network. The idea behind this detection methodology is that the decision to install and activate a new application in a Software Defined network environment is made infrequently. Before activating the new application in production, tests are carried out in an environment where the network operator has complete control of the actions that are performed inside it. By doing so, through the log file produced, it is possible to highlight the malignant or benign nature of the application being tested. In addition to the proposed detection solution, an in-depth security analysis is carried out which revealed security vulnerabilities within the ONOS environment. For the first time in literature, it's implemented a Cross Application Poisoning attack which does not aim to damage the correct functioning of the network, in this case a web application is used as target and the attack presented exploits a Cross Site Scripting vulnerability.
\end{abstract}